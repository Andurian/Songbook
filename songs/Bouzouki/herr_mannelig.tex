\beginsong{Herr Mannelig}[
	by={Traditional}
]

\beginverse
Bittida en morgon innan solen upprann
Innan foglarna började sjunga
Bergatrollet friade till fager ungersven
Hon hade en falskeliger tunga
\endverse

\beginchorus
Herr Mannelig herr Mannelig trolofven i mig
För det jag bjuder så gerna
I kunnen väl svara endast ja eller nej
Om i viljen eller ej
\endchorus

\beginverse
Eder vill jag gifva de gångare tolf
Som gå uti rosendelunde
Aldrig har det varit någon sadel uppå dem
Ej heller betsel uti munnen
\endverse

\beginverse
Eder vill jag gifva de qvarnarna tolf
Som stå mellan Tillö och Ternö
Stenarna de äro af rödaste gull
Och hjulen silfverbeslagna
\endverse

\beginverse
Eder vill jag gifva ett förgyllande svärd
Som klingar utaf femton guldringar
Och strida huru I strida vill
Stridsplatsen skolen i väl vinna
\endverse

\beginverse
Eder vill jag gifva en skjorta så ny
Den bästa I lysten att slita
Inte är hon sömnad av nål eller trå
Men virkat av silket det hvita
\endverse

\beginverse
Sådana gåfvor jag toge väl emot
Om du vore en kristelig qvinna
Men nu så är du det värsta bergatroll
Af Neckens och djävulens stämma
\endverse

\beginverse
Bergatrollet ut på dörren sprang
Hon rister och jämrar sig svåra
Hade jag fått den fager ungersven
Så hade jag mistat min plåga
\endverse

\endsong