\beginsong{Blau wie das Meer}[
	by={Mr.~Hurley \& die Pulveraffen}
]

\begin{infotext}
Mit Instrumentalbegleitung:
	\begin{itemize}
		\item Intro zweimal spielen
		\item Refrain ohne Gesang wiederholen
	\end{itemize}
\end{infotext}

\beginverse*
\[C] \[F] \[Am] \[G]
\[C] \[F] \[Am] \[G] \[C] \[G] \[C]
\endverse

\beginverse\memorize
Schon als \[Am]Schiffsjunge hab ich meine \[F]Seele ver\[G]kauft
für ne \[F]große Buddel Rum mit 3 \[G]X-en darauf
Ich \[Am]will nur kurz dran nippen, da pas\[F]siert mir ein Mal\[G]heur:
Der \[F]Korken fällt mir über Bord, die \[G]Flasche muss leer!
\endverse

\beginchorus
Ich bin \[C]blau wie das Meer \[F]voll wie unser Laderaum
\[Am]breit so wie die Ärsche von den \[G]Weibern auf Tortuga
Ich bin \[C]blau wie das Meer ge\[F]laden wie ein Bordgeschütz
und \[Am]dichter als der \[G]Nebel vor Kap \[C]Ho-\[G]o-\[C]orn
\endchorus

\beginverse
Der ^Schiffsarzt sagt mir jeden Tag ich ^tränke zu viel ^Rum
er ^bangt um meine Leber, appel^liert an die Vernunft
Doch ^wär für uns das Wasser zum ^trinken ge^dacht
^hätte Gott den Ozean nicht ^salzig gemacht!
\endverse

\newpage

\textnote{Gesprochen und verkatert}

\beginverse
^Gestern Abend habe ich wohl ^ein' zu viel ge^habt
ich ^wache auf und hab in meiner ^Koje wenig Platz
ich ^drehe mich nach Steuerbord und ^was muss ich da ^seh'n?
In ^meinem Bett liegt nackt die Frau ^vom \[G]Ka\[G]pi\[G]tän
\endverse

\textnote{Refrain normal ohne Wiederholung}

\beginchorus
Sie war \[C]blau wie das Meer \[F]voll wie unser Laderaum
\[Am]breit so wie die Ärsche von den \[G]Weibern auf Tortuga
Sie war \[C]blau wie das Meer ge\[F]laden wie ein Bordgeschütz
und \[Am]dichter als der \[G]Nebel vor Kap \[C]Ho-\[G]o-\[C]orn
\endchorus

\textnote{Halftime}
\beginverse
Und \[F]kann ich mich morgens noch \[C]daran erinnern \[G]wo ich einge\[C]schlafen bin
\[F]muss das Gelage wohl \[C]trostlos gewesen \[G]sein 
Wir \[F]liegen viel länger im \[C]Seemannsgrab als \[G]dass wir lebendig \[C]sind
also \[F]gießt den drei Ma\[C]trosen noch einen \[G]ein
\endverse

\textnote{Refrain ohne Begleitung}
\textnote{Refrain}

\beginverse*
\[C] \[F] \[Am] \[G]
\[C] \[F] \[Am] \[G] \[C] \[G] \[C]
\endverse

\endsong