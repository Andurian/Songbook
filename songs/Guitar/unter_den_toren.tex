\beginsong{Unter den Toren}[
	by={Traditional}
]

\begin{infotext}
Angenehmer zu singen, wenn die Gitarre einen Ganzton tiefer gestimmt ist.
\end{infotext}

\beginverse*
\[Dm] \[C] \[F] \[A]
\endverse

\beginverse\memorize
\[Dm]Unter den Toren, im \[C]Schatten der Stadt, 
schläft man \[B]gut, wenn man sonst keine \[A]Schlafstelle hat. 
\[Dm]Keiner, der \[C]fragt nach Wo\[F]her und Wo\[G]hin, 
und zu \[Dm]kalt ist die \[Am]Nacht für Gen\[Dm]darmen. 
\endverse

\beginchorus
\lrep \[Dm]He \[C]ho ein \[F]Feuerlein \[C]brennt, 
\[Dm]kalt ist die \[Am]Nacht für Gen\[Dm]darmen. \rrep \rep{2}
\[Dm] \[C] \[F] \[A]
\endchorus

\beginverse
^Silberne Löffel und ^Ketten im Sack, 
legst du ^besser beim Schlafen dir ^unters Genack. 
^Zeig nichts und ^sag nichts, die ^Messer sind ^stumm, 
und zu ^kalt ist die ^Nacht für Gen^darmen.
\endverse 

\beginverse
^Greif nach der Flasche, doch ^trink nicht zuviel, 
deine ^Würfel sind gut, aber ^falsch ist das Spiel.
^Guck in die ^Asche und ^schau lieber ^zu, 
denn zu ^kalt ist die ^Nacht für ^Gendarmen. 
\endverse

\beginverse
^Rückt dir die freundliche ^Schwester zu nach,
es ist ^nur für die Wärme, mal ^hier und mal da.
^Keiner im ^Dunkeln ver^liert sein ^Gesicht,
denn zu ^kalt ist die ^Nacht für Gen^darmen.
\endverse

\beginverse
^Geh mit der Nacht, eh der ^Frühnebel steigt, 
nur das ^Feuer brennt stumm und das ^Steinpflaster schweigt.
^Laß nichts zu^rück und ver^giß, was du ^sahst, 
denn die ^Sonne bringt ^bald die Gen^darmen.
\endverse

\beginchorus
\lrep \[Dm]He \[C]ho das \[F]Feuer ist \[C]aus, 
\[Dm]bald kommen \[Am]die Gen\[Dm]darmen. \rrep \rep{2}
\endchorus
\endsong